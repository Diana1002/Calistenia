\documentclass{article}
\usepackage[utf8]{inputenc}
\usepackage[spanish]{babel}
\usepackage{listings}
\usepackage{graphicx}
\graphicspath{ {images/} }
\usepackage{cite}

\begin{document}

\begin{titlepage}
    \begin{center}
        \vspace*{1cm}
            
        \Huge
        \textbf{Parcial 1 }
            
        \vspace{0.5cm}
        \LARGE
        Calistenia
            
        \vspace{1.5cm}
            
        \textbf{DIANA LUCIA BAEZA RUIZ}
            
        \vfill
            
        \vspace{0.8cm}
            
        \Large
        Despartamento de Ingeniería Electrónica y Telecomunicaciones\\
        Universidad de Antioquia\\
        Medellín\\
        Marzo de 2021
            
    \end{center}
\end{titlepage}

\tableofcontents
\newpage
\section{Sección introductoria}\label{intro}
A continuación, hago una descripción través de una serie de pasos de como pasar de una posición inicial a una posicin final. Sobre una mesa se encuentran dos tarjetas, una sobre otra y encima de ellas una hoja de papel. La tarea consiste en que luego de ejecutar los pasos descritos a continuacion las tarjetas pasen a estar encima de la hoja de papel, formando una piramide en forma de triangulo, utilizando una sola mano. 

\section{Pasos a seguir}\label{contenido}
Tener en cuenta que los pasos a continucación deben realizarse utilizando una sola mano.
\begin{enumerate}
    \item Colocamos encima de la mesa dos tarjetas juntas(de igual peso), una encima de la otra y ambas en la misma orientacion. Encima de las tarjetas una hoja de papel.

    \item Vamos a rodar la hoja de papel. La quitamos de encima de las tarjetas y las colocamos a un lado. 

    \item Tomamos las tarjetas en la mano.

    \item Apoyamos las tarjetas sobre la hoja de papel.

    \item Sin levantar el brazo de la mesa.

    \item Formar un triángulo con las dos tarjetas, apoyandose una sobre la otra de los lados que tengan menos medidas,  hasta que se sostengan solas.
    
    

\end{enumerate}


\subsection{Citación}

Para mayor ilustración se me hace necesario mostrar el siguiente video, en donde se evidencia el paso a paso de la solución del problema.

[1] YouTube. \url{https://youtu.be/MxJvFZuDeqE}


   


\end{document}
